%=====================================================================================================

%       Math 244 Lecture Notes

%       Chapters 22 Day One

%				201701

%				LaTeX=>PDF sizing, graphics
%=====================================================================================================


\documentclass[12pt]{amsart}
\usepackage{amssymb,latexsym}
\usepackage{graphicx}
\usepackage{multicol}
\usepackage{multirow}
\usepackage{amsmath}
\usepackage{calc}
\usepackage{enumitem}
\usepackage{fancyhdr,lastpage}
\usepackage{setspace}
\usepackage{caption}
\usepackage{framed}
\usepackage{wasysym}

\theoremstyle{definition}
\pagestyle{fancy}
\newtheorem{theorem}{Theorem}
\newtheorem{lemma}{Lemma}
\newtheorem{defi}{Definition}
\newtheorem*{notation}{Notation}
\newtheorem{ex}{Example}
\newtheorem{proc}{Process}
\newtheorem{gw}{Group Work}
\newtheorem*{summary}{}

\setlength{\textwidth}{7.25in}			%for =>PDF, use 6.5in
\setlength{\textheight}{9.5in}		%for =>PDF, use 9.5in
\setlength{\oddsidemargin}{-0.375in}		
\setlength{\evensidemargin}{-0.375in}		
\setlength{\voffset}{-.75in}			%for=>PDF, use -0.75in
\setlength{\footskip}{15pt}


% For degree symbol
\newcommand{\degree}{\ensuremath{^\circ}}



% enumerate settings
\setlist{itemsep=4pt}	%topsep=0.1em
\renewcommand{\labelenumi}{ (\alph{enumi}) }			% makes enumeration (a) instead of (a).



\usepackage{placeins} % enables \FloatBarrier, useful for positioning figures/tables more precisely.
%\usepackage[dvips]{graphics}
%\usetikzlibrary{arrows}


\usepackage{pgfplots}

\tikzset{>=stealth}
\usetikzlibrary{backgrounds,arrows}
\tikzset{tight background}


\pgfplotsset{every axis/.append style={width=6.75cm,height=6.75cm,
					axis x line=middle,
					axis y line=middle,
					line width=0.75pt,
					tick label style={font=\tiny},
					label style={font=\small},
					legend style={font=\tiny}
}}

\pgfplotsset{framed/.style={axis background/.style ={draw=black!75}}}

\pgfmathdeclarefunction{gauss}{3}{%
	\pgfmathparse{1/(#3*sqrt(2*pi))*exp(-((#1-#2)^2)/(2*#3^2))}%
}



\usepackage{hyperref}
%\usepackage{url}
\hypersetup{colorlinks=true,urlcolor=blue,linkcolor=blue,breaklinks=true}

\usepackage{xcolor}
%		My colors
\definecolor{silver}{rgb}{0.95,0.95,0.95}
\definecolor{mypurple}{cmyk}{0.5,1,0,0}			%purple
\definecolor{myblue}{cmyk}{1,0.012,0,0}		%blue
\definecolor{mygreen}{cmyk}{1,0,0.99995,0}		%green
\definecolor{myred}{cmyk}{0,1,1,0}


\setlength{\parindent}{0pt}

\date{}



\begin{document}

\newcommand{\ph}{\phantom}
\newcommand{\ds}{\displaystyle}

\renewcommand{\emph}{\textbf}
\onehalfspace

%============================================================
%           Header Stuff   CHANGE FOR EACH SECTION!!!
%============================================================

\fancyhf{}   % clears both header and footer
\fancyfoot[RE,RO]{ \scriptsize{Page \thepage\ of \pageref{LastPage}}}
\fancyfoot[LE,LO]{\scriptsize{Instructor:  J.Wherry}}
\fancyhead[RE,RO]{\scriptsize{Chapter 22: Comparing Two Proportions }}
\fancyhead[LE,LO]{\scriptsize{Math 244 Lecture Notes }}
\renewcommand{\headrulewidth}{0.4pt} % Removes header line if 0pt
\fancyfootoffset[LE,LO]{0in}        %Moves center ??
\renewcommand{\footrulewidth}{0.4pt} % Removes header line if 0pt



%============================================================
%           Title/ Info
%============================================================

\begin{center}

	\larger[3]	Math 244 Lecture Notes \smaller[3]		\\[22pt]

\end{center}

\section*{Chapter 22 Day One: Comparing Two Proportions Using a Confidence Interval}




 \textbf{Overview:} Today, we will compare the success rates/proportions for two groups using a confidence interval. This will use many of the techniques from the previous chapters.
 ~\\
 
 \begin{framed}
  We found our confidence interval for one-proportion using the formula $$\hat{p}\pm Z^*\sqrt{\frac{\hat{p}\hat{q}}{n}}$$ or rather (Center)$\pm$(Dist$)^*$SE. Confidence intervals are based off \underline{\hspace{1in}}
 \end{framed}
 
\begin{framed}
We used the model in step two of a hypothesis test of $$\hat{p}=N\left(p,\sqrt{\frac{pq}{n}} \right).$$ Hypothesis tests are based off \underline{\hspace{1in}}.
\end{framed}

Both our CI and our H-Test were created from the same model. So, if we have a model for comparing two proportions, we'll be able to test claims and determine the actual difference in two groups.\\
~\\
\begin{ex} Suppose that $P_{Seattle}-P_{Portland}>0$ for two winning rates for local roller derby teams. Which city is better?\end{ex} \vfill

\begin{ex} Suppose that $P_{Seattle}-P_{Portland}<0$ for two winning rates for local roller derby teams. Which city is better?\end{ex} \vfill

\begin{ex} Suppose that $P_{Seattle}-P_{Portland}=0$ for two winning rates for local roller derby teams. Which city is better?\end{ex} \vfill

NOTE:
\newpage
We need a model for $\hat{p}_1-\hat{p}_2$. We start with some quick observations...
\begin{enumerate}
 \item The model for a single proportion is $$\hat{p}_1=N\left(p_1,\sqrt{\frac{p_1q_1}{n_1}} \right).$$
 \item If you add/subtract two normal models, you will get another normal model.
 \item If $X$ and $Y$ are independent variables, then $$Var[X-Y]=Var[X]+Var[Y].$$
\end{enumerate}

Our model for $\hat{p}_2-\hat{p}_1$ is normal based off the first two facts. Let's determine which normal.

\begin{ex} CENTER: $E[\hat{p}_1]=p_1$ and $E[\hat{p}_2]=p_2$ are the centers for each of our individual models. What is the center for $E[\hat{p}_1-\hat{p}_2]$?
\end{ex}

\vspace{1in}

\begin{ex} VARIANCE: $Var[\hat{p}_1]=\frac{p_1q_1}{n_1}$ and $Var[\hat{p}_2]=\frac{p_2q_2}{n_2}$ are the variances for each of our individual models. What is the variance for the difference, $Var[\hat{p}_1-\hat{p}_2]$?
\end{ex}

\vspace{1.5in}

\begin{ex}Based off the variance, what is the standard deviation, $SD[\hat{p}_1-\hat{p}_2]$?
\end{ex}
\vspace{0.5in}
\begin{framed}
 Our model is
 
 \vspace{0.75in}
 
\end{framed}

\newpage
\begin{ex}
Using the model $$\hat{p}_1-\hat{p}_2=N\left( p_1-p_2, \sqrt{\frac{p_1q_1}{n_1}+\frac{p_2q_2}{n_2}}\right)$$ create the confidence interval for $p_1-p_2$.\\
HINT: Our interval will be based off samples and look something like (Center)$\pm Z^*$ SE.
\end{ex}

\vspace{1in}

Our assumptions are...
\begin{enumerate}
 \item \,
 \item \,
 \item \, 
 \item \,
 \vspace{1in}
\end{enumerate}

\begin{ex} Create a $90\%$ CI for $p_1-p_2$ if $x_1=18$, $n_1=25$, $x_2=22$, and $n_2=50$.
\end{ex}

\vspace{2in}

Which group appears to have a larger success rate? Why?

\vspace{.25in}

\newpage
\begin{ex} Mr. Wherry performs a random study of PCC students and finds that out of $200$ students, under the age of $20$, $120$ have played Pok\'{e}mon Go! In contrast, he finds that out of $100$ students, age $20$ or older, $40$ have played Pok\'{e}mon Go!
 Create a $95\%$ for the difference in proportions for Pok\'{e}mon Go! players for the two age groups. Assume that people were sampled individually to avoid bias.\end{ex}
 
 \vfill
 
 \vfill
 
 \begin{ex} Suppose you got a $95\%$ CI for $P_{Seattle}-P_{Portland}$ of $(-0.04,-0.03)$. What does this tell us?
 \end{ex}
 
 \vfill
 
 \begin{ex} Suppose you got a $95\%$ CI for $P_{Seattle}-P_{Portland}$ of $(-0.04,0.10)$. What does this tell us?
 \end{ex}
 
 \vfill
 
 \begin{ex} Suppose you got a $95\%$ CI for $P_{Seattle}-P_{Portland}$ of $(-0.04,-0.03)$. What would the $95\%$ for $P_{Portland}-P_{Seattle}$ look like?
 \end{ex}
 
 \vfill

 \newpage

 Batman Preference: Create a $95\%$ CI to examine the difference in proportions for people that like Batman.
 \vspace{3in}
 
\begin{ex} What was the best point estimate in the previous problem?\end{ex}

\begin{ex} What was the MOE in the previous problem?\end{ex}
~\\
\hrule
~\\
\noindent \textbf{Checking work with Calculator:} In our calculator, we use ``2-propZInt" to check our interval. This is found in either [Stat]$\rightarrow$[Tests] on the TI-83/84 OR [Stat/List]$\rightarrow$[F7:Ints] on the TI-89. Type in your relevant information and you are good to go!\\
~\\
Calc Interval:
\newpage
\begin{ex}
	Mike and his friend El are playing a game of Dungeons and Dragons. Mike thinks El might be cheating by moving the die. In $50$ rolls, Mike gets $4$ ``critical rolls". El gets $14$ ``critical rolls" in $60$ rolls. Determine if El is getting more ``critical hits" using a $95\%$ confidence interval.
\end{ex}

\begin{ex}
	Jenny and Dory are competing to see who can remember more street names. Dory managed to remember $40\%$ of $150$ tested street names. Jenny remembered $60\%$ of $150$ tested street names. Determine if Dory remembers less street names using a $99\%$ confidence interval.
\end{ex}

\begin{ex}
	Jack and Diane, two American kids growing up in the heartland, are competing to see how many Slurpee flavors they can correctly identify. Jack identified $10$ correct out of $30$. Diane identified $13$ correct out of $30$. Determine if the difference is statistically significant using a $90\%$ confidence interval.
\end{ex}

\begin{ex}
	Mr. Wherry has two cats: Victor and Fiona. In the $200$ selected afternoons, Fiona is at the front door waiting $180$ times. In contrast, Victor is at the front door waiting $40$ times in a different $50$ randomly selected days. Determine if Fiona is more frequently at the front door waiting using a $99\%$ confidence interval.
\end{ex}
\end{document}
