%=====================================================================================================

%       Math 244 Lecture Notes

%       Chapter 26 Day Two: Analysis of Variation

%				201701

%				LaTeX=>PDF sizing, graphics
%=====================================================================================================


\documentclass[12pt]{amsart}
\usepackage{amssymb,latexsym}
\usepackage{graphicx}
\usepackage{multicol}
\usepackage{multirow}
\usepackage{amsmath}
\usepackage{calc}
\usepackage{enumitem}
\usepackage{fancyhdr,lastpage}
\usepackage{setspace}
\usepackage{caption}
\usepackage{framed}
\usepackage{wasysym}
\usepackage{pst-func}

\theoremstyle{definition}
\pagestyle{fancy}
\newtheorem{theorem}{Theorem}
\newtheorem{lemma}{Lemma}
\newtheorem{defi}{Definition}
\newtheorem*{notation}{Notation}
\newtheorem{ex}{Example}
\newtheorem{proc}{Process}
\newtheorem{gw}{Group Work}
\newtheorem*{summary}{}

\setlength{\textwidth}{7.25in}			%for =>PDF, use 6.5in
\setlength{\textheight}{9.5in}		%for =>PDF, use 9.5in
\setlength{\oddsidemargin}{-0.375in}		
\setlength{\evensidemargin}{-0.375in}		
\setlength{\voffset}{-.75in}			%for=>PDF, use -0.75in
\setlength{\footskip}{15pt}


% For degree symbol
\newcommand{\degree}{\ensuremath{^\circ}}



% enumerate settings
\setlist{itemsep=4pt}	%topsep=0.1em
\renewcommand{\labelenumi}{ (\alph{enumi}) }			% makes enumeration (a) instead of (a).



\usepackage{placeins} % enables \FloatBarrier, useful for positioning figures/tables more precisely.
%\usepackage[dvips]{graphics}
%\usetikzlibrary{arrows}


\usepackage{hyperref}
%\usepackage{url}
\hypersetup{colorlinks=true,urlcolor=blue,linkcolor=blue,breaklinks=true}

\usepackage{xcolor}
%		My colors
\definecolor{silver}{rgb}{0.95,0.95,0.95}
\definecolor{mypurple}{cmyk}{0.5,1,0,0}			%purple
\definecolor{myblue}{cmyk}{1,0.012,0,0}		%blue
\definecolor{mygreen}{cmyk}{1,0,0.99995,0}		%green
\definecolor{myred}{cmyk}{0,1,1,0}


\setlength{\parindent}{0pt}

\date{}



\begin{document}

\newcommand{\ph}{\phantom}
\newcommand{\ds}{\displaystyle}

\renewcommand{\emph}{\textbf}
\onehalfspace

%============================================================
%           Header Stuff   CHANGE FOR EACH SECTION!!!
%============================================================

\fancyhf{}   % clears both header and footer
\fancyfoot[RE,RO]{ \scriptsize{Page \thepage\ of \pageref{LastPage}}}
\fancyfoot[LE,LO]{\scriptsize{Instructor:  J.Wherry}}
\fancyhead[RE,RO]{\scriptsize{Chapter 26 Day Two: Analysis of Variation (ANOVA)}}
\fancyhead[LE,LO]{\scriptsize{Math 244 Lecture Notes }}
\renewcommand{\headrulewidth}{0.4pt} % Removes header line if 0pt
\fancyfootoffset[LE,LO]{0in}        %Moves center ??
\renewcommand{\footrulewidth}{0.4pt} % Removes header line if 0pt



%============================================================
%           Title/ Info
%============================================================

\begin{center}

	\larger[3]	Math 244 Lecture Notes \smaller[3]		\\[22pt]

\end{center}

\section*{Chapter 26 Day Two: Analysis of Variation (ANOVA)}




 \textbf{Overview:} Last class, we learned how to compare multiple groups. Today, we will look at terminology more critically and determine how to do ANOVA with technology\\
 ~\\
Here is the general process from start to finish:
\begin{framed}
STEP I: State the hypotheses. $H_0:\mu_1=...=\mu_k$ and $H_A:$ At least one average is different.

STEP II: When comparing multiple averages, we use the $F$-Distribution. Where...

$$\text{Variation Between Groups}=\Sigma \frac{n_i(\bar{x}_i-\bar{X})^2}{k-1}$$ with $k$ the number of groups and $\bar{X}$ the overall average AND
$$\text{Variation Within Groups}=s_{pool}^2=\Sigma \frac{df_i(s_i)^2}{N-k}$$ with $N$ the total number of values.
$$F=\frac{\text{Variation Between Groups}}{\text{Variation Within Groups}}$$ has df's of $k-1$ and $N-k$ in that order.\\
~\\
The assumptions are
\begin{enumerate}
 \item Random
 \item Independence/$10\%$ Rule
 \item Large n:
 \item Indepedent Groups
 \item Equal Spread:
 
 \vspace{0.5in}
\end{enumerate}

STEP III: ANOVA is always a right-tailed test.\\ We find the P-value using ``Fcdf($F$-score, $\infty$, $df_{Between}$,$df_{Within}$)''

STEP IV: Conclusion

\end{framed}

\begin{ex} Perform ANOVA for the Excel file located on your desktop.
\end{ex}

\vfill
\newpage
We've covered the neccessary math at this point, but let's improve our vocabulary.
\begin{framed}
 \emph{Variation Between Groups} in a medical setting might compare the average duration of various \emph{treatments}.
 $$\text{Variation Between Groups}=\frac{\Sigma n_i(\bar{x}_i-\bar{X})^2}{k-1}=\frac{\text{Sum of the Squares of the Treatment}}{df}$$
 By dividing the Sum of the Squares of the Treatment (SST), we are finding the Mean for the Squares of the Treatment (MST).
 $$\text{Variation Between Groups}=\frac{\text{SST}}{df}=\text{MST}$$
\end{framed}

\begin{framed}
 \emph{Variation Within Groups} is based solely off the Standard \emph{Errors}.
 $$\text{Variation Within Groups}=s_{pool}^2=\Sigma \frac{df_i(s_i)^2}{N-k}=\frac{\text{Sum of the Squared standard Errors}}{df}$$
 By dividing the Sum of the Squared standard Errors (SSE), we are finding the Mean for the Squares of the Errors (MSE). This is our \emph{pooled} variance.
 $$\text{Variation Between Groups}=s_{pool}^2=\frac{\text{SSE}}{df}=\text{MSE}$$
\end{framed}


\begin{framed}Using the terminology above, we rewrite to get that
 $$F=\frac{\text{Variation Between Groups}}{\text{Variation Within Groups}}=\frac{\text{MST}}{\text{MSE}}$$
\end{framed}

OBSERVATION: You may have noticed this already, but $$df_{top}+df_{bottom}=(k-1)+(N-k)=N-1.$$ This will come in handy for the next exercise.\\
~\\
\newpage
\begin{ex}
 Fill in the following ANOVA table:
 
 {%
\newcommand{\mc}[3]{\multicolumn{#1}{#2}{#3}}
\begin{center}
\begin{tabular}{|c|c|c|c|cc}\hline
Source & df & SS & MS & \mc{1}{c|}{F-Stat} & \mc{1}{c|}{P-Value}\\\hline
Treatment & 2 & 28.22 & × & \mc{1}{c|}{×} & \mc{1}{c|}{×}\\\hline
Error & 24 & 28.44 & × & × & ×\\\cline{1-4}
Total & 26 & 56.66 & × & × & ×\\\cline{1-4}
 \end{tabular}
 \end{center}
}%

\end{ex}

\vspace{1.4in}

\noindent Verifying our answers in the calculator takes two processes. We first must type in our lists. Then, we must perform the test.\\
~\\
\noindent \textbf{Creating lists:} If you have a TI-83, TI-84, hit [Stat]$\rightarrow$[Edit] your lists. Type in your dat.\\
~\\
If you have a TI-89, go into [Apps]$\rightarrow$[Stat/List Editor]. Then, type in your data.\\
~\\
\noindent \textbf{Checking H-Test with Calculator:} In our calculator, we use ``ANOVA" (1-way not 2-way) to check our hypothesis test. This is found in either [Stat]$\rightarrow$[Tests] on the TI-83/84 OR [Stat/List]$\rightarrow$[F6:Tests] on the TI-89. Type in your relevant information and you are good to go!\\
~\\
It will look like ``ANOVA($L_1,L_2,L_3$)'' on the TI-83/84. To get $L_1$, hit [2nd]$\rightarrow$[1]. To get $L_2$, hit [2nd]$\rightarrow$[2] and so on.\\
In the TI-89, you will have to type in the complete name of the lists. For example ``list1''.\\
~\\
\begin{ex} Use your calculator to collect the following information for the warm-up data on tip amounts.\end{ex}
\begin{itemize}
 \item MST?
 \item MSE?
 \item F?
 \item P-value?
 \item Pooled Standard Error? $s_p$?
\end{itemize}

\newpage
\noindent What do we do if we do notice a difference in averages? Follow it up with $2$-sample $t$-tests. The downside? Error adds up really quickly.\\
\begin{ex}
 If we want to compare three groups, how many pairs will we need to analyze?
\end{ex}

\vspace{0.5in}

\begin{ex}
 If we want the total $\alpha$ for all tests to be $0.05$, what would each individual $\alpha^*$ need to be?
\end{ex}
\vspace{1in}

\begin{framed} 
 Bonferroni Correction. Let $$\alpha^*=$$ when comparing multiple pairs. This will give you a corrected $C^*=1-\alpha*$ if you want to create confidence intervals as a follow up.
\end{framed}

\begin{ex} Compare each pair of individuals for the tip data. Which pairs, if any, indicate a difference in average tip amount?
\end{ex}
\vfill
\newpage
\textbf{Options, Options, Options:} We now have multiple ways of doing problems involving averages.
\begin{ex}
You want to compare two routes to work. Route 1 takes $30$ minutes on average with a standard deviation of $15$ minutes and $n=100$. In contrast, Route 2 takes $25$ minutes on average with a standard deviation of $10$ minutes and $n=100$ [When both sample sizes are the same size the study is \emph{balanced}]. Compare the averages using ANOVA.
\end{ex}
\vfill

\begin{ex}
You want to compare two routes to work. Route 1 takes $30$ minutes on average with a standard deviation of $15$ minutes and $n=100$. In contrast, Route 2 takes $25$ minutes on average with a standard deviation of $10$ minutes and $n=100$. Compare the averages using 2-sampletTest. Go ahead and pool.\end{ex}
\vfill
OBSERVATIONS:
\vspace{1in}

\end{document}
