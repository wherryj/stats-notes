%=====================================================================================================

%       Math 244 Lecture Notes

%       Chapter 24 Chi-Squared Goodness of Fit

%				201701

%				LaTeX=>PDF sizing, graphics
%=====================================================================================================


\documentclass[12pt]{amsart}
\usepackage{amssymb,latexsym}
\usepackage{graphicx}
\usepackage{multicol}
\usepackage{multirow}
\usepackage{amsmath}
\usepackage{calc}
\usepackage{enumitem}
\usepackage{fancyhdr,lastpage}
\usepackage{setspace}
\usepackage{caption}
\usepackage{framed}
\usepackage{wasysym}
\usepackage{pst-func}

\theoremstyle{definition}
\pagestyle{fancy}
\newtheorem{theorem}{Theorem}
\newtheorem{lemma}{Lemma}
\newtheorem{defi}{Definition}
\newtheorem*{notation}{Notation}
\newtheorem{ex}{Example}
\newtheorem{proc}{Process}
\newtheorem{gw}{Group Work}
\newtheorem*{summary}{}

\setlength{\textwidth}{7.25in}			%for =>PDF, use 6.5in
\setlength{\textheight}{9.5in}		%for =>PDF, use 9.5in
\setlength{\oddsidemargin}{-0.375in}		
\setlength{\evensidemargin}{-0.375in}		
\setlength{\voffset}{-.75in}			%for=>PDF, use -0.75in
\setlength{\footskip}{15pt}


% For degree symbol
\newcommand{\degree}{\ensuremath{^\circ}}



% enumerate settings
\setlist{itemsep=4pt}	%topsep=0.1em
\renewcommand{\labelenumi}{ (\alph{enumi}) }			% makes enumeration (a) instead of (a).



\usepackage{placeins} % enables \FloatBarrier, useful for positioning figures/tables more precisely.
%\usepackage[dvips]{graphics}
%\usetikzlibrary{arrows}


\usepackage{hyperref}
%\usepackage{url}
\hypersetup{colorlinks=true,urlcolor=blue,linkcolor=blue,breaklinks=true}

\usepackage{xcolor}
%		My colors
\definecolor{silver}{rgb}{0.95,0.95,0.95}
\definecolor{mypurple}{cmyk}{0.5,1,0,0}			%purple
\definecolor{myblue}{cmyk}{1,0.012,0,0}		%blue
\definecolor{mygreen}{cmyk}{1,0,0.99995,0}		%green
\definecolor{myred}{cmyk}{0,1,1,0}


\setlength{\parindent}{0pt}

\date{}



\begin{document}

\newcommand{\ph}{\phantom}
\newcommand{\ds}{\displaystyle}

\renewcommand{\emph}{\textbf}
\onehalfspace

%============================================================
%           Header Stuff   CHANGE FOR EACH SECTION!!!
%============================================================

\fancyhf{}   % clears both header and footer
\fancyfoot[RE,RO]{ \scriptsize{Page \thepage\ of \pageref{LastPage}}}
\fancyfoot[LE,LO]{\scriptsize{Instructor:  J.Wherry}}
\fancyhead[RE,RO]{\scriptsize{Chapter 24: Chi-Squared Goodness of Fit}}
\fancyhead[LE,LO]{\scriptsize{Math 244 Lecture Notes }}
\renewcommand{\headrulewidth}{0.4pt} % Removes header line if 0pt
\fancyfootoffset[LE,LO]{0in}        %Moves center ??
\renewcommand{\footrulewidth}{0.4pt} % Removes header line if 0pt



%============================================================
%           Title/ Info
%============================================================

\begin{center}

	\larger[3]	Math 244 Lecture Notes \smaller[3]		\\[22pt]

\end{center}

\section*{Chapter 24: Chi-Squared Goodness of Fit}




 \textbf{Overview:} We return today to analyzing counts and proportions, much like the material from Exam I. Today, however, we will find a way to analyze multiple counts all at once.\\
 ~\\
 \noindent Mr. Wherry is given a ``fair'' six sided die. He's not convinced in its fairness however, and after rolling it several times he gets the following outcomes:
 
\begin{center}
\begin{tabular}{c|c|c|c|c|c|c}
Outcome of Die & 1 & 2 & 3 & 4 & 5 & 6\\\hline
$\#$ of trials & 9 & 13 & 9 & 12 & 8 & 15
\end{tabular}
\end{center}
\vspace{1in}

\begin{ex} 
\begin{enumerate}
\item If the die is actually fair, what proportion of the time should Mr. Wherry get an outcome of a ``1''? How many counts should he expect for the number of ``1''s? Do this for all values.

\vspace{0.5in}

\item In terms of counts, how far off is the observed number of trials (table) from the expected number of trials (last part)?

\vspace{0.5in}

\item What is the relative/percentage error for each outcome? For example, maybe the observed outcome $2$ is $50\%$ more than what we were expecting [not actual numbers].

\vfill

\item Should we use a test or an interval to determine if the die is fair?
\vspace{0.5in}
\end{enumerate}
\end{ex}

NOTE: Our total for the relative/percentage errors for each outcome is \underline{\hspace{0.25in}}.

\newpage
\textbf{What happened?!}

\vspace{0.25in}


\noindent Let's create a new model: $$\chi^2=$$

\vspace{0.5in}

\noindent Degrees of freedom are based off the number of groups. For example, our die problem has df$=\underline{\hspace{1in}}$. In general, df$=$.\\
~\\
\begin{framed}
 \noindent The assumptions are...
\begin{enumerate}
 \item Countable data.
 \item \,
 \item \,
 \item \,
\end{enumerate}
\end{framed}
\vspace{0.2in}

To help you visualize it, $\chi^2$ looks like the graph below. The \emph{mean=df} and the mound is at df-2.

\psset{xunit=0.5cm,yunit=20cm,arrowscale=1.5}
\begin{pspicture}(-1,-0.1)(21,0.2)
\psChiIIDist[linewidth=1pt,nue=5]{0.01}{19.5}
\psaxes[labels=none,ticks=none]{->}(20,0.2)
%\pscustom[fillstyle=solid,fillcolor=red!30]{%
%\psChiIIDist[linewidth=1pt,nue=5]{8}{19.5}%
%\psline(20,0)(8,0)}
\end{pspicture}


\newpage
\noindent 
\begin{ex}
Let's look at the six-sided die hypothesis test in detail. Recall that           
\begin{center}
\begin{tabular}{c|c|c|c|c|c|c}
Outcome of Die & 1 & 2 & 3 & 4 & 5 & 6\\\hline
OBSERVED $\#$ of trials & 9 & 13 & 9 & 12 & 8 & 15\\ \hline
EXPECTED $\#$ of trials & & & & & &
\end{tabular}
\end{center}
\vspace{0.1in}

\noindent STEP I:
\vspace{0.2in}

\noindent STEP II:
\vfill

\noindent STEP III:
\vfill

\noindent STEP IV:
\vspace{0.5in}
\end{ex}

\newpage
\noindent \textbf{Which way do we shade?}\\
\noindent Assume that the die is perfectly fair...
\begin{enumerate}
 \item How should the observed and expected compare?
 \item Should we have a large value for $\chi^2$ or a small value?
 \item Do we want to reject or fail to reject the null-hypothesis?
 \item Do we want a large or small P-value?
 \item SHADE IN THE CURVE:

 \psset{xunit=0.5cm,yunit=15cm,arrowscale=1.5}
\begin{pspicture}(-1,-0.1)(21,0.2)
\psChiIIDist[linewidth=1pt,nue=5]{0.01}{19.5}
\psaxes[labels=none,ticks=none]{->}(20,0.2)
%\pscustom[fillstyle=solid,fillcolor=red!30]{%
%\psChiIIDist[linewidth=1pt,nue=5]{8}{19.5}%
%\psline(20,0)(8,0)}
\end{pspicture}

\end{enumerate}
\noindent Assume that the die is extremely unfair...
\begin{enumerate}
 \item How should the observed and expected compare?
 \item Should we have a large value for $\chi^2$ or a small value?
 \item Do we want to reject or fail to reject the null-hypothesis?
 \item Do we want a large or small P-value?
 \item SHADE IN THE CURVE:

 \psset{xunit=0.5cm,yunit=15cm,arrowscale=1.5}
\begin{pspicture}(-1,-0.1)(21,0.2)
\psChiIIDist[linewidth=1pt,nue=5]{0.01}{19.5}
\psaxes[labels=none,ticks=none]{->}(20,0.2)
%\pscustom[fillstyle=solid,fillcolor=red!30]{%
%\psChiIIDist[linewidth=1pt,nue=5]{8}{19.5}%
%\psline(20,0)(8,0)}
\end{pspicture}

\end{enumerate}

\noindent OBSERVATION:
\vspace{0.2in}

\emph{TI-83 and TI-84:} [2nd]$\rightarrow$[Vars]$\rightarrow$[chi-square cdf].\\
\emph{TI-89:} [Apps]$\rightarrow$[Stat/List]$\rightarrow$[F5:Dist]$\rightarrow$[chi-square cdf].\\
~\\
The general setup is the same for all calculators. We need to type in ``chisquarecdf(low $\chi^2$-score, high $\chi^2$-score, df)''.\\
Our P-value$=$\underline{\hspace{5in}}
\newpage
\begin{ex} We're going to analyze a bag of M$\&$M's. Let's test the claim that the color distribution is uniform. The colors are blue, brown, green, orange, red, and yellow.
\end{ex}
\begin{center}
\begin{tabular}{c|c|c|c|c|c|c}
 & Blue& Brown& Green& Orange& Red& Yellow\\\hline
OBSERVED $\#$ of trials &  &  &  &  &  & \\ \hline
EXPECTED $\#$ of trials & & & & & &
\end{tabular}
\end{center}
\vspace{0.1in}

\noindent STEP I:
\vspace{0.2in}

\noindent STEP II:
\vfill

\noindent STEP III:
\vfill

\noindent STEP IV:
\vspace{0.5in}

\begin{framed}
 We call $Obs-Exp$ the residual.
 
 We call $\frac{Obs-Exp}{\sqrt{Exp}}$ the standardized residual. It looks like the square-root of a single entry for our $\chi^2$ and it works like a z-score!
\end{framed}

\begin{ex} Calculate all the standardized residuals.
\end{ex}
\begin{center}
\begin{tabular}{c|c|c|c|c|c|c}
 & Blue& Brown& Green& Orange& Red& Yellow\\\hline
St. Res. &  &  &  &  &  & \\
\end{tabular}
\end{center}
\vspace{0.3in}

\newpage
\noindent \textbf{Critical Value:} Just like $Z^*$ and $t^*$, we can find $(\chi^2)^*$ using a table. This can be thought of as the cutoff for what we consider a large or small $\chi^2$-score.\\
~\\
\begin{ex} Let's reanalyze the bag of M$\&$M's. This time, we will test the claim for the company itself. They claim that the percentages are as follows: blue$=24\%$, brown$=13\%$, green$=16\%$, orange$=20\%$, red$=13\%$, and yellow$=14\%$.
\end{ex}
\begin{center}
\begin{tabular}{c|c|c|c|c|c|c}
 & Blue& Brown& Green& Orange& Red& Yellow\\\hline
OBSERVED $\#$ of trials &  &  &  &  &  & \\ \hline
EXPECTED $\#$ of trials & & & & & &
\end{tabular}
\end{center}
\vspace{0.1in}

\noindent STEP I:
\vspace{0.2in}

\noindent STEP II:
\vfill

\noindent STEP III:
\vfill

\noindent STEP IV:
\vspace{0.5in}

\begin{ex} Recalculate all the standardized residuals.
\end{ex}
\begin{center}
\begin{tabular}{c|c|c|c|c|c|c}
 & Blue& Brown& Green& Orange& Red& Yellow\\\hline
St. Res. &  &  &  &  &  & \\
\end{tabular}
\end{center}
\vspace{0.3in}

\end{document}
