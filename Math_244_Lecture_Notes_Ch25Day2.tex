%=====================================================================================================

%       Math 244 Lecture Notes

%       Chapters 25 Day Two

%				201701

%				LaTeX=>PDF sizing, graphics
%=====================================================================================================


\documentclass[12pt]{amsart}
\usepackage{amssymb,latexsym}
\usepackage{graphicx}
\usepackage{multicol}
\usepackage{multirow}
\usepackage{amsmath}
\usepackage{calc}
\usepackage{enumitem}
\usepackage{fancyhdr,lastpage}
\usepackage{setspace}
\usepackage{caption}
\usepackage{framed}
\usepackage{wasysym}

\theoremstyle{definition}
\pagestyle{fancy}
\newtheorem{theorem}{Theorem}
\newtheorem{lemma}{Lemma}
\newtheorem{defi}{Definition}
\newtheorem*{notation}{Notation}
\newtheorem{ex}{Example}
\newtheorem{proc}{Process}
\newtheorem{gw}{Group Work}
\newtheorem*{summary}{}

\setlength{\textwidth}{7.25in}			%for =>PDF, use 6.5in
\setlength{\textheight}{9.5in}		%for =>PDF, use 9.5in
\setlength{\oddsidemargin}{-0.375in}		
\setlength{\evensidemargin}{-0.375in}		
\setlength{\voffset}{-.75in}			%for=>PDF, use -0.75in
\setlength{\footskip}{15pt}


% For degree symbol
\newcommand{\degree}{\ensuremath{^\circ}}



% enumerate settings
\setlist{itemsep=4pt}	%topsep=0.1em
\renewcommand{\labelenumi}{ (\alph{enumi}) }			% makes enumeration (a) instead of (a).



\usepackage{placeins} % enables \FloatBarrier, useful for positioning figures/tables more precisely.
%\usepackage[dvips]{graphics}
%\usetikzlibrary{arrows}


\usepackage{pgfplots}

\tikzset{>=stealth}
\usetikzlibrary{backgrounds,arrows}
\tikzset{tight background}


\pgfplotsset{every axis/.append style={width=6.75cm,height=6.75cm,
					axis x line=middle,
					axis y line=middle,
					line width=0.75pt,
					tick label style={font=\tiny},
					label style={font=\small},
					legend style={font=\tiny}
}}

\pgfplotsset{framed/.style={axis background/.style ={draw=black!75}}}

\pgfmathdeclarefunction{gauss}{3}{%
	\pgfmathparse{1/(#3*sqrt(2*pi))*exp(-((#1-#2)^2)/(2*#3^2))}%
}



\usepackage{hyperref}
%\usepackage{url}
\hypersetup{colorlinks=true,urlcolor=blue,linkcolor=blue,breaklinks=true}

\usepackage{xcolor}
%		My colors
\definecolor{silver}{rgb}{0.95,0.95,0.95}
\definecolor{mypurple}{cmyk}{0.5,1,0,0}			%purple
\definecolor{myblue}{cmyk}{1,0.012,0,0}		%blue
\definecolor{mygreen}{cmyk}{1,0,0.99995,0}		%green
\definecolor{myred}{cmyk}{0,1,1,0}


\setlength{\parindent}{0pt}

\date{}



\begin{document}

\newcommand{\ph}{\phantom}
\newcommand{\ds}{\displaystyle}

\renewcommand{\emph}{\textbf}
\onehalfspace

%============================================================
%           Header Stuff   CHANGE FOR EACH SECTION!!!
%============================================================

\fancyhf{}   % clears both header and footer
\fancyfoot[RE,RO]{ \scriptsize{Page \thepage\ of \pageref{LastPage}}}
\fancyfoot[LE,LO]{\scriptsize{Instructor:  J.Wherry}}
\fancyhead[RE,RO]{\scriptsize{Chapter 25: Linear Regression Day 2}}
\fancyhead[LE,LO]{\scriptsize{Math 244 Lecture Notes }}
\renewcommand{\headrulewidth}{0.4pt} % Removes header line if 0pt
\fancyfootoffset[LE,LO]{0in}        %Moves center ??
\renewcommand{\footrulewidth}{0.4pt} % Removes header line if 0pt



%============================================================
%           Title/ Info
%============================================================

\begin{center}

	\larger[3]	Math 244 Lecture Notes \smaller[3]		\\[22pt]

\end{center}

\section*{Chapter 25 Day One: Linear Regression Day 2}




 \noindent \textbf{Overview:} Linear Regression is used for examining the relationship between two variables--an \textbf{explanatory variable}, x, and a \textbf{response variable}, y.\\
 \begin{framed}When done correctly, it allows us to make predictions about what should happen given a limited amount of information. Here's what we know so far.
\begin{itemize}
\item Our approximation for the line of best fit is $\hat{y}=b_1(x)+b_0$ where $b_1$ is our slope and $b_0$ is our $y$-intercept.
\item We calculate the slope using the formula $b_1=r\frac{s_y}{s_x}$.
\item We found the slope by plugging in the point $(\bar{x},\bar{y})$ and solving for $b_0$.
\item $r$ and $R^2$ were both calculated using technology. Recall that $R^2$ is the percentage of variation accounted for by our model.
\item Residuals are the error in our predictions. We find these by $e=y-\hat{y}=\text{data value}-\text{line value}.$
\end{itemize}\end{framed}
\begin{framed}
 
Much to our surprise, we found that under ideal situations the residuals followed a t-distribution $$e\approx t_{k-2}(0,s_e)$$ where $s^2_e=\Sigma\frac{(e-0)^2}{k-2}$.\\

\end{framed}
Unfortunately, error is unavoidable. We are making two big predictions that cause this. We predict a value for the slope $b_1$ when the actual slope is $\beta_1$. We predict a value for the starting value $b_0$ when the actual starting value is $\beta_0$. As a result, we ``lost" two degrees of freedom due to the unknown factors. This gave us $df=k-2$. Both of these predictions are affected by our residual distribution, so both of these unknowns can be analyzed with the t-distribution. The assumptions for these tests include...
\begin{itemize}
\item The usual: Random, Independence.
\item Linearity: The data should look linear when plotted. Of course.
\item Equal Spread: Much like ANOVA. This means that the data stays clumped near the line.
\item Normal Population for Errors: We want the errors to look normal. Everything else is based off this. We can either do a histogram of the errors to check for nearly normal OR do a normality plot.
\end{itemize}
\newpage\noindent Let's do one final experiment for the term! I want to know if the drop height of a ball in centimeters affects the bounce height of a ball.

\begin{center}
\begin{tabular}{|c|c|}
\hline
Drop Height & Bounce Height \\
\hline
10 & \\
\hline
15 & \\
\hline
20 & \\
\hline
25 & \\
\hline
30 & \\
\hline
35 & \\
\hline
40 & \\
\hline
45 & \\
\hline
50 & \\
\hline
55 & \\
\hline
60 & \\
\hline
65 & \\
\hline
70 & \\
\hline
75 & \\
\hline
80 & \\
\hline
85 & \\
\hline
\end{tabular}
\begin{enumerate}
\item State the Explanatory and Response Variable.
\vspace{0.3in}
\item Determine the line of best fit. Using technology is okay.
\vspace{0.3in}
\item Calculate $R^2$. How good is the fit?
\vspace{0.3in}
\item \textbf{Test for Linearity} Using technology, perform the test for linearity! Recall that this refers to $\beta_1$.
\vfill
\item Assumption Check:
\begin{enumerate}
\item The usual: 
\item Linearity: 
\item Equal Spread: 
\item Normal Population for Errors: 
\vspace{0.5in}
\end{enumerate}
\newpage
\noindent We will use a printout from CrunchIt \url{<http://crunchit2.bfwpub.com/crunchit2/ips5e/?section_id=>} for the last two components. Fill in the table first. You may be expected to read such tables for the final.

\begin{center}
\begin{tabular}{|c|c|c|c|c|}
\hline
Estimates &  t-Stat & P-Value & CI Low & CI High\\
\hline
(Intercept)& & &  \\
\hline
Drop Height& & & \\
\hline
\end{tabular}
\end{center}
\item \textbf{Test for $y$-intercept, $\beta_0$.} Using the printout, perform a test for a $y$-intercept. State your hypotheses, degrees of freedom, model, P-value, and conclusion.
\vfill
\vfill
\item State the $95\%$ CI for the slope, $\beta_1$.
\vfill
\item State the  $95\%$ CI for the y-intercept.
\vfill
\item Create the $95\%$ CI for predicted bounce height if we drop the ball from $100$ cm.
\vfill
\end{enumerate}
\end{center}
\end{document}
